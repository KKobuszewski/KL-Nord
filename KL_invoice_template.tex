%% LyX 2.3.4.2 created this file.  For more info, see http://www.lyx.org/.
%% Do not edit unless you really know what you are doing.
\documentclass[oneside]{amsbook}
\usepackage[utf8]{inputenc}
\usepackage{geometry}
\geometry{verbose,tmargin=0.75cm,bmargin=0.75cm,lmargin=1.0cm,rmargin=1.0cm}
\setlength{\parindent}{0bp}
\usepackage{booktabs}
\usepackage{amsthm}
\usepackage[unicode=true,pdfusetitle,
 bookmarks=true,bookmarksnumbered=false,bookmarksopen=false,
 breaklinks=false,pdfborder={0 0 1},backref=false,colorlinks=false]
 {hyperref}
\usepackage{multicol}

\makeatletter

%%%%%%%%%%%%%%%%%%%%%%%%%%%%%% LyX specific LaTeX commands.
%% Because html converters don't know tabularnewline
\providecommand{\tabularnewline}{\\}

%%%%%%%%%%%%%%%%%%%%%%%%%%%%%% Textclass specific LaTeX commands.
\numberwithin{section}{chapter}
\numberwithin{equation}{section}
\numberwithin{figure}{section}

\makeatother

\begin{document}
%\begin{tabular}{llllll}
\begin{tabular*}{0.9\linewidth}{@{\extracolsep{\fill}} llllll}
Orginal / Copy  &  &  &  & data rachunku  & {{ date }} \tabularnewline
Invoice / Faktura VAT  & No.: {{ klno }} &  &  & invoice date  & \tabularnewline
 &  &  &  &  & \tabularnewline
 &  &  &  & data sprzedaży & {{ date }} \tabularnewline
 &  &  &  & sales date & \tabularnewline
\end{tabular*}
% forcing a table into page width
%\begin{tabular*}{\linewidth}{@{\extracolsep{\fill}} llllll}
%	Orginal / Copy  &  &  &  & data rachunku  & ???\tabularnewline
%	Invoice / Faktura VAT  & No.:  & ??? &  & invoice date  & \tabularnewline
%	&  &  &  &  & \tabularnewline
%	&  &  &  & data sprzedaży & ???\tabularnewline
%	&  &  &  & sales date & \tabularnewline
%\end{tabular*}

\textbf{Nabywca / BUYER }

{{ bname }}

2-4 Mesogeion Ave.

Athens GR 115 27 Greece

VAT (EU) No.: {{ vatno }}

~\\

\textbf{Sprzedajacy / SELLER}

\textbf{KL Nord Lech Kobuszewski}

PL 76-270 Przewloka, ul. Galczynskiego 16

VAT (EU) No.: PL 839 170 44 35

~\\
\begin{multicols}{2}
Bank: PKO BP Oddz.1 Slupsk, ul. Wazów 5

IBAN: PL11102046490000790201167295

KOD SWIFT: BPKOPLPW

\columnbreak

PŁATNOŚĆ: 60 DAYS FROM THE DATE

PAYMENT: BY SWIFT TRANSFER
\end{multicols}

\begin{tabular*}{\linewidth}{@{\extracolsep{\fill}}|ll|ll|ll|}
\hline 
\multicolumn{1}{l}{} & \multicolumn{1}{l}{} &  & \multicolumn{1}{l}{} &  & \multicolumn{1}{l}{}\tabularnewline
\hline 
l.P. USŁUGA  &  & WARTOŚĆ NETTO &  & WARTOŚĆ BRUTTO & \tabularnewline
agency servise  &  & NET VALUE &  & GROSS VALUE & \tabularnewline
\hline 
 &  &  &  &  & \tabularnewline
NAZWA &  & EUR &  & EUR & \tabularnewline
DESCRIPTION &  &  &  &  & \tabularnewline
\hline 
 &  &  &  &  & \tabularnewline
PROWIZJA AGENCYJNA &  &  &  &  & \tabularnewline
zgodnie z &  &  &  &  & \tabularnewline
fakturą nr {{ invno }} &  &  &  &  & \tabularnewline
{{ mass }}~t x {{ eurpert}} EUR/t &  &  &  &  & \tabularnewline
\cline{3-6} \cline{4-6} \cline{5-6} \cline{6-6} 
 &  &  &  &  & \tabularnewline
AGENCY COMMISSION &  &  &  &  & \tabularnewline
acc. to invoice &  &  &  &  & \tabularnewline
No. {{ invno }} fr {{ data }} &  &  & {{ value }} &  & {{ value }}\tabularnewline
\hline 
\multicolumn{1}{|l|}{} & \multicolumn{1}{l}{RAZEM: } &  &  &  & \tabularnewline
\multicolumn{1}{|l|}{Adnotacje:} & \multicolumn{1}{l}{TOTAL: } &  & {{ value }} &  & {{ value }}\tabularnewline
\cline{2-6} \cline{3-6} \cline{4-6} \cline{5-6} \cline{6-6} 
\multicolumn{1}{|l|}{Annotations:} & \multicolumn{1}{l}{W TYM:} &  &  &  & \tabularnewline
\multicolumn{1}{|l|}{} & \multicolumn{1}{l}{INCL.:} &  & {{ value }} &  & {{ value }}\tabularnewline
\cline{2-6} \cline{3-6} \cline{4-6} \cline{5-6} \cline{6-6} 
\multicolumn{1}{|l|}{} & \multicolumn{1}{l}{} &  &  &  & \tabularnewline
\multicolumn{1}{|l|}{} & \multicolumn{1}{l}{} &  &  &  & \tabularnewline
\cline{2-6} \cline{3-6} \cline{4-6} \cline{5-6} \cline{6-6} 
\multicolumn{1}{|l|}{} & \multicolumn{1}{l}{} &  &  &  & \tabularnewline
\multicolumn{1}{|l|}{} & \multicolumn{1}{l}{} &  &  &  & \tabularnewline
\hline 
\end{tabular*}

~\\

\textbf{DO ZAPŁATY: {{ value }}~EUR}

\textbf{TO BE PAID:}

~\\

SŁOWNIE: {{ valuesay }}~EUR

SAY:

~\\

SERVICE'S BUYER IS THE VALUE ADDED TAX PAYER

Nabywca usługi jest płatnikiem podatku VAT „odwrotne obciążenie”

~\\

Signature

Kobuszewski Lech
\end{document}
